%!TEX root = ../report.tex

\chapter{Introduction}

The ease with which the humans and animals accomplish extremely complex motions has influenced the research in robotic motion planning and control at various levels, right from the motor control to high the level planning. It is widely accepted that humans learn a great variety of movements, and that these movements are stored in some form in our memories \cite{lim2005movement}. One key observation can be made that the biological motions are consisted of \textit{motion primitives} (basic units of motion) perfected through experience over time\cite{schaal2006dynamic}. This can be concluded from the example of a tennis player. A tennis player takes months of practice to perfect his \textit{move} and to learn when to use it as well. This example is just a representative of vast number of skills acquired by humans and animals through experience. Various efforts have been made to adopt the concept of such motion primitives to generate robust robot control policy. Many variants of motion primitives are summarized in \cite{kober2013reinforcement} and \cite{deisenroth2013survey} which are necessarily model-free motion planning approaches because the primitives are learned independently without considering robot and environment model and validated at the time of execution. A motion primitive framework built around second order differential equations representing mass-spring damped system called \textit{Dynamic Movement Primitives (DMP)} is particularly famous and this work uses the same.  

\par Dynamic motion primitive is essentially an \textit{Learning from Demonstration} approach. A trajectory in the joint space or the task space is obtained from human demonstration. Then the control policy behind that trajectory is learned in the attractor space of nonlinear dynamic equations.

\par An alternative for above mentioned biological skills is model-based motion planning and model-based control policy search. Both of these need a fairly accurate model of robot as well as the environment which is hard to obtain. Need of skills and experience required for motion execution is replaced by the accurate model of robot and environment. 

\par In this project, a learning from demonstration framework using dynamic movement primitives was implemented on KUKA YouBot mobile manipulator and Toyota HSR. Various experiments were performed to prove the usability of DMPs in RoboCup@Work and RoboCup@Home scenarios and identify the need for knowledge base for DMPs. 

While conducting the experiments, limitations on executing the trajectories generated by DMPs were revealed which triggered the idea of combining mobile base motion with manipulator motion for tracking the trajectories which lead to the development of whole body motion control framework. Whole body motion significantly enhanced the manipulation capabilities of both the robots.   
